\documentclass[11pt]{article}

\usepackage[margin=3cm]{geometry}
\usepackage{amsmath}

\title{The Optimal Strategy for the Game Lucky Numbers}
\author{Isaac Riley}
\date{\today}

\begin{document}
\maketitle
\section{Game Rules}
In the game \textit{Lucky Numbers}, wo to four players compete for 
\section{Formal Representation}
For each player, we calculate the range of permissible numbers for each empty square. 
For example %%% EXAMPLE HERE %%%
Each empty square $s^{(k)}_{ij}$ is assigned a number $r^{k}_{ij}$ representing the number of distinct cards that are allowed at in that position.
These numbers form the set $R^{(k)}$.
Moreover, define $b^{(k)}$ as the number of blank squares remaining on the board. 

For each turn, we can define a full vector $\textbf{v}^{(k)}$ of indicators, as follows:\\
\begin{tabular}{cl}
Indicator & Description \\ \hline
$v_1$ & change in number of empty spots (0 or 1) \\
$v_2$ & change in minimum $r^{(k)}_{ij}$ \\
$v_3$ & change in maximum $r^{(k)}_{ij}$ \\
$v_4$ & change in mean $r^{(k)}_{ij}$ \\
$v_5$ & change in median $r^{(k)}_{ij}$ \\
$v_6$ &  \\
$v_7$ & 
\end{tabular}
\section{Results}

\section{Discussion}
\end{document}